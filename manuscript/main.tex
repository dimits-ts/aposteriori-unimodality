\documentclass{article}
\usepackage{graphicx}
\usepackage{amsmath}
\usepackage{acronym}

\title{Aposteriori Unimodality}
\author{Dimitris }
\date{March 2025}


% Insert space after commas in math mode
\AtBeginDocument{%
  \mathchardef\stdcomma=\mathcode`,
  \mathcode`,="8000
}
\begingroup\lccode`~=`, \lowercase{\endgroup\def~}{\stdcomma\,}

\begin{document}

\maketitle

\section{Methodology}

\subsection{Intuition}

Let's say 

\subsection{Problem Formulation}

We define a conversation $d \in D$ as an ordered set of comments\footnote{Also referred to as “dialogue turns” in some publications.}:
\begin{equation}
    d = \{c(d, 1), c(d, 2), \ldots, c(d, \lvert d \rvert)\}
\end{equation}


We assume that annotating a comment depends on three variables: the content of the comment, the annotator \ac{SDB} (represented as $\theta \in \Theta$), and uncontrolled factors such as mood and personal experiences. Since each comment is likely to be annotated by a distinct subset of annotators, we define $S_{d, i} \subseteq \Theta$ as the set of annotators responsible for annotating the comment $c(d, i)$. We can then define the set of annotations as:

\begin{equation}
    A = \{a(d, i, \theta) \mid i=1, 2, \ldots, \lvert d \rvert, \theta \in S_{d, i} \}
\end{equation}

Since our goal is to pinpoint which specific characteristics contribute to polarization, we need a way to isolate and analyze individual attributes within a \ac{SDB}. We define $\theta$ as a structured composition of multiple attributes, where each attribute corresponds to a specific demographic or personal trait, and each attribute has an associated value. More formally, we can express $\theta$ as a set of feature-factor pairs:

\begin{equation}
    \theta = \{(\xi_i, \mu) \mid i=1, 2, \ldots, k \mathpunct{,} \mu \in M_i\}
\end{equation} 
\noindent where $\Xi=\{\xi_1, \xi_2, \ldots, \xi_k\}$ is the set of features (e.g., age, gender, education level, political affiliation), and $M_i$ is the set of possible factors for attribute $\xi_i$ (e.g., if $\xi_1$ corresponds to gender, then $M_1=\{\textit{male}, \textit{female}, \ldots\}$).


\subsection{The Aposteriori Unimodality Test}

Define the partition $P(A,\xi, \mu)$ as the set of annotations $A$ for a comment $c(d, i)$ for which the \ac{SDB} $\theta$ includes the feature-factor pair $(\xi, \mu)$. In other words,
\begin{equation}
    P(A, \xi, \mu) = \{a(d, i, u) \in A | (\xi, \mu) \in \theta\}
\end{equation}

Focusing on the annotations for a single comment, we introduce the “r-statistic” to measure the distance from unimodality. Suppose the feature of interest is $\xi_j$. Then, for comment $c(d, i)$, the distance from unimodality is defined as:

\begin{equation}
    r(d, i, \xi_j) = \max_{\mu \in M_j}\{nDFU(A) - nDFU(P(A, \xi_j, \mu)\}
\end{equation}

\noindent where $A$ denotes the complete set of annotations associated with comment $c(d, i)$. Intuitively, $r$ quantifies how much the observed polarization can be attributed to the feature $\xi_j$, with higher values of $r$ indicating that a significant part of the polarization is due to variations in $\xi_j$. It is obvious that $r \rightarrow [0,1]$.

The reason $r$ is computed on the comment-level instead over the whole discussion is because each comment depends on a number of confounding variables, including previous comments, its content, its intended recipients, etc. However, annotations referring to each individual comment control for most of these variables, allowing us to isolate the effect of $\xi$ on annotator polarization.

If the polarization in a discussion$d$ is not driven by the feature $\xi_j$, we would expect $r(d, i, \xi_j) \approx 0, \forall i=1, 2, \ldots$. Consequently, even in the face of limited samples and random noise, we can apply a non-parametric t-test with the null hypothesis $H_0: \sum_{i=1}^k r(d, i, \xi_j) = 0$, versus the alternative hypothesis $H_a: \sum_{i=1}^k r(d, i, \xi_j) > 0$. This procedure is referred to as the Aposteriori Unimodality Test, where a small p-value suggests that we can not rule out that the feature $\xi_j$ makes a significant contribution to the overall annotator polarization.


\section{Acronyms}

\begin{acronym}[WWW]
    \acro{SDB}{SocioDemographic Background}
\end{acronym}

\end{document}
