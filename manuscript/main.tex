\documentclass{article}
\usepackage{graphicx}
\usepackage{amsmath}
\usepackage{acronym}

\title{Aposteriori Unimodality}
\author{Dimitris }
\date{March 2025}


% Insert space after commas in math mode
\AtBeginDocument{%
  \mathchardef\stdcomma=\mathcode`,
  \mathcode`,="8000
}
\begingroup\lccode`~=`, \lowercase{\endgroup\def~}{\stdcomma\,}



\begin{document}

\maketitle

\section{Methodology}

\subsection{Intuition}

Let's say 


\subsection{Problem Formulation}
\label{ssec:methodology:problem}

We define a conversation $d \in D$ as an ordered set of comments\footnote{Also referred to as “dialogue turns” in some publications.}:
\begin{equation}
    d = \{c(d, 1), c(d, 2), \ldots, c(d, \lvert d \rvert)\}
\end{equation}

We assume that annotating a comment depends on three variables: its contents, the annotator's \ac{SDB} (represented as $\theta \in \Theta$), and uncontrolled factors such as mood and personal experiences. Since each comment is likely to be annotated by a distinct subset of annotators, we define $S_{d, i} \subseteq \Theta$ as the set of annotators responsible for annotating the comment $c(d, i)$. We can then define the set of annotations as:
\begin{equation}
    A = \{a(d, i, \theta) \mid i=1, 2, \ldots, \lvert d \rvert, \theta \in S_{d, i} \}
\end{equation}

Since our goal is to pinpoint which specific characteristics contribute to polarization, we need a way to isolate individual attributes within a \ac{SDB}. By definition, $\theta$ is composed of multiple ``dimensions'' (e.g., age, sex, educational level). We can thus model $\theta$ as a set of attribute-value pairs, where each attribute corresponds to a specific socio-demographic trait, and is associated  with at most one value. More formally:
\begin{equation}
    \theta = \{(\xi_i, \mu) \mid i=1, 2, \ldots, k \mathpunct{,} \mu \in M_i\}
\end{equation} 
\noindent where $\Xi=\{\xi_1, \xi_2, \ldots, \xi_k\}$ is the set of features or ``dimensions'' and $M_i$ is the set of possible values for attribute $\xi_i$ (e.g., if $\xi_1$ corresponds to gender, then $M_1=\{\textit{male}, \textit{female}, \ldots\}$).


\subsection{The Aposteriori Unimodality Test}
\label{ssec:methodology:aposteriori}

The mechanism defined in Section \ref{ssec:methodology:problem} allows us to isolate the effects of each \ac{SDB} dimension, but we still lack a mechanism with which to analyze that effect. In this section, we define the ``Aposteriori Unimodality test'' which will allows us to determine whether polarization in a comment can be attributed to a specific \ac{SDB} attribute $\xi$.

Intuitively, $\xi$ contributes to polarization in an annotation set when the annotations split by  $\xi$ are more polarized compared to the entire set of annotations. However, using all annotations in a discussion is not ideal because we can not control for factors like different comment content. For example, if women find one comment toxic more than men, and another much less toxic than men, the opposing effects might balance each other out, leading to a false negative. To avoid this, we apply our statistic only within the same comment’s annotations instead. Thus, we can define our "pol-statistic" (polarization statistic) as:
\begin{equation}
	pol(d, i, \mu) = nDFU(A) - nDFU(P(A, \xi, \mu))
\end{equation}
\noindent where $P(A,\xi, \mu) = \{a(d, i, u) \in A | (\xi, \mu) \in \theta\}$ is the set of annotations $A$ for a comment $c(d, i)$ for which the \ac{SDB} $\theta$ includes the attribute-value pair $(\xi, \mu)$.  Intuitively, $pol \rightarrow [0,1]$ quantifies how much the observed polarization can be attributed to the value $\mu$ of attribute $\xi$.

Although intuitive, the $pol$ statistic can only be applied to individual comments, and is susceptible to inherent noise present in annotation tasks. If the polarization in a discussion $d$ is not driven by the attribute $\xi$, we would expect $pol(d, i, \mu) \approx 0, \forall i=1, 2, \ldots, \lvert d \rvert, \forall \mu \in M_{\xi}$. Consequently, even in the face of limited samples and random noise, we can apply a mean test with the null hypothesis $H_0: \sum_{i=1}^k pol(d, i, \mu) = 0 \forall \mu \in M_{\xi}$, versus the alternative hypothesis $H_a: \exists \mu \in M_{\xi}: \sum_{i=1}^k pol(d, i, \mu) > 0$. Since we are considering $\lvert M_{\xi} \rvert$ tests, we apply a bonferroni correction to the resulting p-values.

This procedure is referred to as the \textit{``Aposteriori Unimodality Test''}, where a small p-value suggests that we can not rule out that the feature $\xi$ makes a significant contribution to the overall annotator polarization.


\section{Acronyms}

\begin{acronym}[WWW]
    \acro{SDB}{SocioDemographic Background}
\end{acronym}

\end{document}
